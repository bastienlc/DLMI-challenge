\documentclass{midl}
\usepackage{mwe}
\usepackage[english]{babel}

\title[DLMI]{Lymphocytosis Classification Challenge}

\midlauthor{\Name{Joachim COLLIN} \Email{joachim.collin@eleves.enpc.fr}
\AND
\Name{Bastien LE CHENADEC} \Email{bastien.le-chenadec@eleves.enpc.fr}
}

\begin{document}

\maketitle

\begin{abstract}
\end{abstract}

\section{Introduction}
\label{sec:introduction}

Lymphocytosis is a common hematologic abnormality characterized by an increase in the absolute concentration of lymphocytes to more than 4000 lymphocytes/microL for adult patients \cite{Hamad_2023}. This condition can arise from various sources, including reactions to infections, drugs, or stress, or it may indicate a lymphoproliferative disorder, which is a type of cancer involving abnormal proliferation of lymphocytes. Clinicians typically diagnose lymphocytosis by assessing personal data such as medical history, symptoms, medication lists, and through a blood test to measure lymphocyte levels. However, additional tests may be necessary to confirm the cause of lymphocytosis and determine an appropriate treatment plan. Each condition associated with lymphocytosis presents its own set of symptoms and treatment options. While the diagnosis process is efficient, it suffers from poor reproducibility, and the additional tests required can be costly and time-consuming \cite{Sahasrabudhe_2021}. Being able to distinguish more accurately reactive from malignant lymphocytosis patients is challenging and would lead to a better identification of patients requiring additional testing.

\section{Methodology}
\label{sec:methodology}

\section{Evaluation}
\label{sec:evaluation}

\section{Conclusions}
\label{sec:conclusion}

\newpage
\bibliography{bibliography}

\end{document}